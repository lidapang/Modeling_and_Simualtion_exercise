\section{Philosophy of science} 
\begin{enumerate}
	\item
	\item
	\item
	\item
	\item
\end{enumerate}
\section{Climate modeling}
\begin{enumerate}
  \item The author briefly introduces the background of global warming and the potential impact to the reader.
  		He points out that you need virtual models to assess what is happening in the real world better and to 
  		create any forecasts. Here for he  points to a number of scientific models. (eg climate models, model of the Earth) 
  		and points to the advantages and disadvantages of  virtual models.
  \item 
  \item	Physical laws are based on the virtual models and describe best the modeled system. These laws are represented by mathematical equations that must be solved 
  		using numerical methods, and this requires a lot of computing power. In addition, a model have a lot of parameters, which in turn complicates the equations.
  \item Virtual  models may not display the reality correctly, they should serve as a supportive agent in the model and allow the user to simulate the influence of some paramter.
  
\end{enumerate}

\section{Monte carlo simulationubtitle}
The following code calculates the \textbf{mean}  and \textbf{standard deviation} of the dataset. 
In this case the inputfile contains all the necessary data.

\begin{python}[caption={Python code for median and standard deviation}]
import math

daten = open('daten.txt')
#gibt die erste zeile aus
#print(daten.readline()) 

#classvars
mean_sum = 0.0 
stand_sum =0.0
outputFileName = "result.txt"

#formatiere datenset in liste
valueList = []
for zeile in daten:
    valueList.extend(zeile.split(' '))

#berechne median
for value in  valueList:
    val_num = float(value)   
    mean_sum += val_num
    
#median
median = mean_sum / len(valueList)

#berechne deviation
for value in  valueList:
    val_num = float(value)
    stand_sum += (val_num - median)**2
    
#deviation
standart_deviation = math.sqrt(stand_sum/len(valueList))
    
print "mean:               " + str(median)
print "standard deviation: " + str(standart_deviation)

outputFile = open(outputFileName, 'w')
outputFile.write("mean:               " + str(median) + "\n" + "standard deviation: " + str(standart_deviation) )
outputFile.close()
daten.close()
\end{python}
The results are:\\
\textbf{mean:				}   149.155 \\        
\textbf {standard deviation:} 	25.537