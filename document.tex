\documentclass[12pt,german,bibliography=totoc]{scrreprt}   
%BCOR8.25mm seht hier bei f�r den abstand von der linken seite, da diese
% gebunden wird
%%%%%%%%%%%%%%%%%%%%%%%%%%%%%%%%%%%%%%%%%%%%%%%%%%%%%%%%%%%%%%%%%%%%%%%%%
%EINGEBUNDENE PAKETE (immer vor dem document ) 
%%%%%%%%%%%%%%%%%%%%%%%%%%%%%%%%%%%%%%%%%%%%%%%%%%%%%%%%%%%%%%%%%%%%%%%%%
\usepackage{bibgerm}			%fuer aufwendigeres literaturverzeichnis mit themes
%\usepackage{amssymb}			% wichtige Symbole
\usepackage{graphics}			% Einbinden von Graphiken
\usepackage[pdftex]{graphicx}
\usepackage{epstopdf}			%automatischen generieren von pdfs aus eps
\usepackage{array}
\usepackage[centertags]{amsmath} 
\usepackage{amsfonts}			% Schriftart
\usepackage{a4}				% Seitenformat
\usepackage{epsfig}				% Zusaetzliche Graphikbefehle 
\usepackage{floatflt}			% Tabellen
\usepackage{scrpage2}			% useheadings
\usepackage[latin1]{inputenc}	% Umlaute in Windows 
\usepackage{paralist}			% Bessere Enumerate-Umgebungen (Aufzaehlungszeichen
% koennen gewaehlt werden) \usepackage{tabular}			% Tabellen
\usepackage{booktabs}
\usepackage[ngerman]{babel}	
\usepackage{mparhack}		% bessere Randnotizen
\usepackage{setspace}		% eigentlich nur fuer den Titel ..
\usepackage{multirow}		% Row-/Columnspan ..
\usepackage{longtable}		%fuer mehrseitige tabellen
\usepackage{makeidx}		%hiermit kann man ein index erstellen, z.b. fuer
% abkuerngen und sozu
\usepackage[T1]{fontenc}	%erweiterter buchstabensatz mit umlauten 
\usepackage{url}			%um URLs darzustellen
\usepackage{thmbox}			%umgebung fuer theoreme und beispiele
\usepackage{color}
\usepackage{xcolor}
\usepackage{caption}
\usepackage{calc}
\usepackage{floatflt}
\usepackage{float}
\usepackage[procnames]{listings}
\usepackage{textcomp}
\usepackage{setspace}
\usepackage{palatino}
% \setlength{\textwidth}{16.5cm}
% \setlength{\evensidemargin}{-0.5cm}
% \setlength{\oddsidemargin}{0.5cm}
% \setlength{\marginparwidth}{18mm}
% \reversemarginpar
% \let\oldmarginpar\marginpar 
% \renewcommand\marginpar[1]{\-\oldmarginpar[\raggedleft\scriptsize #1]%
% {\raggedright\scriptsize #1}}

 %%%%%%%%%%%%%%%%%%%%%%%%%%%%%%%%%%%%%%%%%%%%%%%%%%%%%%%%%%%%%%%%%%%%%%%%%
%FORMATIERUNG DES LISTINGS 
%%%%%%%%%%%%%%%%%%%%%%%%%%%%%%%%%%%%%%%%%%%%%%%%%%%%%%%%%%%%%%%%%%%%%%%%%
\DeclareCaptionFont{white}{\color{white}}
\DeclareCaptionFormat{listing}{\colorbox{gray}{\parbox{\textwidth-2\fboxsep}{#1#2#3}}}
\captionsetup[lstlisting]{format=listing,labelfont=white,textfont=white}
%SMC-Sprache 


%ACCELEO-Sprache
\DeclareCaptionFormat{figure}{\colorbox{gray}{\parbox{\textwidth-2\fboxsep}{#1#2#3}}}
\captionsetup[figure]{format=figure,labelfont=white,textfont=white}

%%%%%%%%%%%%%%%%%%%%%%%%%%%%%%%%%%%%%%%%%%%%%%%%%%%%%%%%%%%%%%%%%%%%%%%%%
%tABELLENFORMATIERUNG
%%%%%%%%%%%%%%%%%%%%%%%%%%%%%%%%%%%%%%%%%%%%%%%%%%%%%%%%%%%%%%%%%%%%%%%%%
%  neuer  Befehl:  \includegraphicstotab[..]{..}
%  Verwendung  analog  wie  \includegraphics
\newlength{\myx}  %  Variable  zum  Speichern  der  Bildbreite
\newlength{\myy}  %  Variable  zum  Speichern  der  Bildh�he
\newcommand\includegraphicsToTab[2][\relax]{%
%  Abspeichern  der  Bildabmessungen
\settowidth{\myx}{\includegraphics[{#1}]{#2}}%
\settoheight{\myy}{\includegraphics[{#1}]{#2}}%
%  das  eigentliche  Einf�gen
\parbox[c][1.1\myy][l]{\myx}{%
\includegraphics[{#1}]{#2}}%
}%  Ende  neuer  Befehl


%%%%%%%%%%%%%%%%%%%%%%%%%%%%%%%%%%%%%%%%%%%%%%%%%%%%%%%%%%%%%%%%%%%%%%%%%
%METAINFORMATIONEN
%%%%%%%%%%%%%%%%%%%%%%%%%%%%%%%%%%%%%%%%%%%%%%%%%%%%%%%%%%%%%%%%%%%%%%%%%
\usepackage[
	pdftitle={Modeling and Simulation Excercises},
	pdfauthor={Alexander Rimer},
	colorlinks=true,
    linkcolor=black,
    citecolor=black,
    filecolor=black,
    pagecolor=black,
    urlcolor=black]{hyperref} %Metainformationen  

%%%%%%%%%%%%%%%%%%%%%%%%%%%%%%%%%%%%%%%%%%%%%%%%%%%%%%%%%%%%%%%%%%%%%%%%%
%SCHRIFT FUER CAPTIONS AENDERN
%%%%%%%%%%%%%%%%%%%%%%%%%%%%%%%%%%%%%%%%%%%%%%%%%%%%%%%%%%%%%%%%%%%%%%%%%
\setkomafont{caption}{\footnotesize \selectfont}
\setkomafont{captionlabel}{\footnotesize \bfseries}
%%%%%%%%%%%%%%%%%%%%%%%%%%%%%%%%%%%%%%%%%%%%%%%%%%%%%%%%%%
%PYTHON HIGHLIGHTING

\include{code_formatter/pythonlisting}
%%%%%%%%%%%%%%%%%%%%%%%%%%%%%%%%%%%%%%%%%%%%%%%%%%%%%%%%%%

%%%%%%%%%%%%%%%%%%%%%%%%%%%%%%%%%%%%%%%%%%%%%%%%%%%%%%%%%%%%%%%%%%%%%%%%%
%DOKUMENT BEGINN
%%%%%%%%%%%%%%%%%%%%%%%%%%%%%%%%%%%%%%%%%%%%%%%%%%%%%%%%%%%%%%%%%%%%%%%%%
\begin{document}  

 

%Die Uebungen    
%Title 
\begin{figure}
\centering
{\Huge Model Building and Simulation (MODUS)}\\[0.5cm]
{\Huge Exercise 1}\\[0.5cm]
{\Large Alexander Rimer}\\[0.6cm]  
\today
\end{figure}

\section{Philosophy of science} 
\begin{enumerate}
	\item
	\item
	\item 
	\item
	\item
\end{enumerate}
\section{Climate modeling}
\begin{enumerate}
  \item The author briefly introduces the background of global warming and the potential impact to the reader.
  		He points out that you need virtual models to assess what is happening in the real world better and to 
  		create any forecasts. Therefore he  points to a number of scientific models. (eg climate models, model of the Earth) 
  		and points to the advantages and disadvantages of  virtual models. 
  		
  \item With a sensitivity test, a modeler can check, in which way parameters  may influence  the result of the virtual model.
 		Furthermore, modeler can find out how much the parameter variation takes effect on the result. The modeler can thus distinguish
 		important from unimportant parameters and may choose the right ones for the model. 
 		
  \item	Physical laws are based on the virtual models and describe best the modeled system. These laws are represented by mathematical equations that must be solved 
  		using numerical methods, and this requires a lot of computing power. In addition, a model has a lot of parameters, which in turn complicates the equations.
  
  \item Virtual  models may not display the reality correctly, they should serve as a supportive 
  		agent in the model and allow the user to simulate the influence of some paramters.
  		With this adapted model you can model for ex. one specific part of a problem domain, so that the number of parameters is kept small. 
  		In such a model  you can now vary individual parameters and study the output of the model, so that a complete virtual model is not absolutely necessary.
  
\end{enumerate}

\section{Monte carlo simulationubtitle}
The following code calculates the \textbf{mean}  and \textbf{standard deviation} of the dataset. 
In this case the inputfile contains all the necessary data.

\begin{python}[caption={Python code for median and standard deviation}]
import math

daten = open('daten.txt')  

#classvars
mean_sum = 0.0 
stand_sum =0.0
outputFileName = "result.txt"

#formatiere datenset in liste
valueList = []
for zeile in daten:
    valueList.extend(zeile.split(' '))

#berechne median
for value in  valueList:
    val_num = float(value)   
    mean_sum += val_num
    
#median
median = mean_sum / len(valueList)

#berechne deviation
for value in  valueList:
    val_num = float(value)
    stand_sum += (val_num - median)**2
    
#deviation
standart_deviation = math.sqrt(stand_sum/len(valueList))
    
print "mean:               " + str(median)
print "standard deviation: " + str(standart_deviation)

outputFile = open(outputFileName, 'w')
outputFile.write("mean:               " + str(median) + "\n" + "standard deviation: " + str(standart_deviation) )
outputFile.close()
daten.close()
\end{python}
The results are:\\
\textbf{mean:				}   149,155 \\        
\textbf {standard deviation:} 	25,537


\end{document}
